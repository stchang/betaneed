\documentclass{article}

\usepackage{amsthm}
\usepackage{amsmath}
\usepackage{fullpage}
\usepackage{betaneed}
\usepackage{pfsteps}

\begin{document}

\subsubsection*{Syntax (new2)}
\begin{align*}
  \tag{Expressions}
  \e   &= x \mid \lxe \mid \ap{\e}{\e} \\
  \tag{Values}
  \val &= \lxe \\
  \tag{Answers}
  \ans &= \Av \\
  \tag{Answer Contexts}
  \A   &= \hole \mid \ap{\inA{\lx{\A}}}{\e} \\
  \tag{Partial Answer Contexts -- outer}
  \Ap &= \hole \mid \ap{\inA{\Ap}}{\e} \\
  \tag{Partial Answer Contexts -- inner}
  \Am &= \hole \mid \inA{\lx\Am} \\
  \tag{Evaluation Contexts} 
  \E   &= \hole \mid \ap{\E}{\e} \mid \inA{\E}
          \mid \inAp{\ap{\inA{\lx\inAm{\inE{x}}}}{\E}} \\
& \hspace{2.9cm} \inAp{\Am} \in \A
\end{align*}


\subsubsection*{Notions of Reduction (new2)}
\begin{align*}
\inAp{\ap{\inA[_1]{\lx\inAm{\Ex}}}{\Av[_2]}}
\;\;\betaneednr&\;\;
\inAp{\inA[_1]{\inA[_2]{\substx{\inAm{\Ex}}{\val}}}} \\
\inAp{\Am} \in& \A 
\end{align*}

%
\begin{align*}
\step&: \textrm{compatible closure of } \betaneednr \\
\steps&: \textrm{reflexive, transitive closure of } \step \\
\pstep&: \textrm{parallel reduction of } \betaneednr \textrm{ redexes} \\
\srstep &: \textrm{standard reduction relation} \\
&\inE{\e} \srstep \inE{\e'} \textrm{ if } \e \;\betaneednr\; \e' \\
\srsteps &: \textrm{reflexive, transitive closure of } \srstep
\end{align*}

%
%
% Definition: parallel reduction
\begin{definition}[$\pstep$] Parallel reduction 

\begin{tabular}{rcllc}
\psteprulenoif
  { \e }{ \e }
\psteprulenoif
  { \inAp{\ap{\inA[_1]{\lx\inAm{\Ex}}}{\Av[_2]}} }
  { \inAp[']{\inA[_1']{\inA[_2']{\substx{\inAm[']{\Ex[']}}{\val'}}}} }
& if & \multicolumn{3}{l}{ $\inAp{\Am} \in A,\, \inAp[']{\Am'} \in \A,$ } \\
& & \multicolumn{3}{l}
    { $\Ap \pstep \Ap',\, \A_1 \pstep \A_1',\, A_2 \pstep \A_2',\,
       \Am \pstep \Am',\, \E \pstep \E',\, \val \pstep \val'$ } \\
\psteprulenoif
  { \ap{\e_1}{\e_2} }{ \ap{\e_1'}{\e_2'} }
& if & \multicolumn{3}{l}{ $\e_1 \pstep \e_1',\; \e_2 \pstep \e_2'$ } \\
\psteprulenoif
  { \lxe }{ \lxe' }
& if & \multicolumn{3}{l} { $\e \pstep \e'$ }
\end{tabular}
\end{definition}


%
%
% Definition: parallel reduction of contexts
\begin{definition}[$\pstep$ for contexts] Parallel reduction of contexts. 

If all subterms in a context $\E$ parallel reduce, then $\E \pstep \E'$, where
each $\e$ in $\E$ is replaced with $\e'$ in $\E'$, and $\e \pstep \e'$.

\begin{tabular}{ccclc}
$\hole$       & $\pstep$ & $\hole$ \\
$\ap{\inA[_1]{\lx\A_2}}{\e}$ 
              & $\pstep$ & $\ap{\inA[_1']{\lx\A_2'}}{\e'}$ 
  & if $\A_1 \pstep \A_1',\, \A_2 \pstep \A_2',\, \e \pstep \e'$ \\
%% $\ap{\Ap}{\e}$ & $\pstep$ & $\ap{\Ap'}{\e'}$ 
%%   & if $\Ap \pstep \Ap',\, \e \pstep \e'$ \\
$\ap{\inA{\Ap}}{\e}$ & $\pstep$ & $\ap{\inA[']{\Ap'}}{\e'}$
  & if $\A \pstep \A',\, \Ap \pstep \Ap',\, \e \pstep \e'$ \\
%% $\lx\Am$       & $\pstep$ & $\lx\Am'$
%%   & if $\Am \pstep \Am'$ \\
$\inA{\lx\Am}$    & $\pstep$ & $\inA[']{\lx\Am'}$
  & if $\A \pstep \A',\, \Am \pstep \Am'$ \\
$\ap{\E}{\e}$ & $\pstep$ & $\ap{\E'}{\e'}$
  & if $\E \pstep \E',\, \e \pstep \e'$ \\
$\inA{\E}$    & $\pstep$ & $\inA[']{\E'}$
  & if $\A \pstep \A',\, \E \pstep \E'$ \\
$\inAp{\ap{\inA{\lx\inAm{\inE[_1]{x}}}}{\E_2}}$
              & $\pstep$ & 
$\inAp[']{\ap{\inA[']{\lx\inAm[']{\inE[_1']{x}}}}{\E_2'}}$
 & if $\inAp{\Am},\,\Ap \pstep \Ap',\, \A \pstep \A',\, \Am \pstep \Am',\,
       \E_1 \pstep \E_1',\, \E_2 \pstep \E_2'$ \\
\end{tabular}
\end{definition}




\pagebreak



%
% Definition: Standard Reduction Sequences
%
\begin{definition}[$\srseqop$]
Standard reduction sequences $\srseqset$
\begin{itemize}
\item $x \subset \srseqset$
\item If $\srseqe{m}$, then $\srseq{\lxe_1}{\lxe_m}$
\item If $\srseqe{m}$ and $\srseqe[']{n}$
      then $\srseqappend{(\ap{\e_1}{\e_1'})}{(\ap{\e_m}{\e_1'})}
            \srseq{(\ap{\e_m}{\e_2'})}{(\ap{\e_m}{\e_n'})}$
\item If $\srseqe{m}$ and $\e_0 \srstep \e_1$, then $\e_0 \srseqop \srseqe{m}$
\end{itemize}
\end{definition}


%
% Definition: Size function for pstep of terms
%
\begin{definition}[size function $\size{\cdot}$]
\begin{align*}
  \size{\e \pstep \e} &= 0 \\
  \size{ \inAp{\ap{\inA[_1]{\lx\inAm{\Ex}}}{\Av[_2]}} 
           \pstep
         \inAp[']{\inA[_1']{\inA[_2']{\substx{\inAm[']{\Ex[']}}{\val'}}}} }
    &= \size{\Ap \pstep \Ap'} + \size{\A_1 \pstep \A_1'} +
       \size{\inAm{\Ex} \pstep \inAm[']{\Ex[']}} \\
    &\hspace{4mm} + \size{\A_2 \pstep A_2'} + 
     \numfv{x}{\inAm[']{\Ex[']}} \times \size{\val \pstep \val'} + 1 \\
  \size{\apppp{e_1}{e_2} \pstep \apppp{\e_1'}{\e_2'}}
    &= \size{\e_1 \pstep \e_1'} + \size{\e_2 \pstep \e_2'} \\
  \size{\lxe \pstep \lxe'} &= \size{\e \pstep \e'}
\end{align*}
\end{definition}


%
% Definition: size function for pstep of contexts
%
\begin{definition}[size function for contexts]
Size of context is equal to sum of sizes of subcontexts and subterms.
\end{definition}



%
% Curry-Feys Standardization Theorem
%
\begin{theorem}[Curry-Feys Standardization]
$\e \steps \e'$ iff $\srseqe{n}$ s.t. $\e = \e_1$ and $\e' = \e_n$.
\end{theorem}

\begin{proof}
By lemma~\ref{lem:pstepsr}
\end{proof}


%
% Main Standardization Lemma 
% If e => e' and srseq:e',e2'...en', there exists srseq:e1,e2...ep s.t.
% e1 = e and ep = en'
%
\begin{lemma}
\label{lem:pstepsr}
If $\e \pstep \e'$ and $\e' \srseqop \srseq{\e_2'}{\e_n'}$, then there exists
$\srseqconse[_1] \srseq{\e_2}{\e_p}$ s.t. $\e_1 = \e$ and $\e_p = \e_n'$.
\end{lemma}

\begin{proof}
By triple lexicographic induction on 1) length $n$ of given standard reduction
sequence, 2) $\size{\e \pstep \e'}$, and 3) structure of $\e$. Proceed by case
analysis on last step in derivation of $\e \pstep \e'$
\begin{byCases}
  %
  % pstep def #1
  \case[(base)]{\e = \e'} Then $\e_1 = \e = \e'$ and $\e_p = \e_n'$
  %
  % pstep def #2
  \case{\e \pstep \e' \textrm{ by } \pstep def(2)}
  $\e = \inAp{\ap{\inA[_1]{\lx\inAm{\Ex}}}{\inA[_2]{\val}}},$\\
  $\e' = \inAp[']{\inA[_1']{\inA[_2']{\substx{\inAm[']{\Ex[']}}{\val'}}}}$\\
  Let $\e'' = \inAp{\inA[_1]{\inA[_2]{\substx{\inAm{\Ex}}{\val}}}}$ \\
  Since $\e$ is $\betaneednr$ redex,
  $\e \srstep \e'' \pstep\e'$\\
  By IH and size lemma(\ref{lem:size}), there exists $\srseq{\e_2}{e_p}$ s.t.
  $\e_2 = \e''$ and $\e_p = \e_n'$
  %
  %pstep def #3
  \case{\e \pstep \e' \textrm{ by } \pstep def(3)}
  $e = \ap{e_3}{e_4}$, $e' = \ap{e_3'}{e_4'}$, 
  $e_3 \pstep e_3'$, $e_4 \pstep e_4'$
  \begin{byCases}
  \renewcommand{\byCasesCaseTemplate}[1]{\textbf{Subcase {#1}.}}
    \case{\e' \srstep \e_2'} 
    By lemma~\ref{lem:srstep}, IH, and reasoning from redex text.
    \case{otherwise}
    Same reasoning as redex text.
  \end{byCases}
  %
  %pstep def #4
  \case{\e \pstep \e' \textrm{ by } \pstep def(4)}
  Claim holds by IH.
\end{byCases}
\end{proof}





\pagebreak


%
% Main Standardization Helper Lemma: convert pstep + srstep to srsteps + pstep
% If e => e* |--> e** there exists e*** s.t. e |-->> e*** => e**
%
\newcommand{\eone}{\e^{*}}
\newcommand{\etwo}{\e^{**}}
\newcommand{\ethree}{\e^{***}}

\begin{lemma}
\label{lem:srstep}
If $\e \pstep \eone \srstep \etwo$, then there exists $\ethree$ s.t. $\e
\srsteps \ethree \pstep \etwo$.
\end{lemma}

\begin{proof}
By double lexicographic induction on $\size{\e \pstep \eone}$ and structure of
$\e$. Proceed by cases on proof of $\e \pstep \eone$.
\begin{byCases}
  %
  % \pstep def(1)
  \case{\e \pstep \eone \textrm{ by } \pstep def(1)} 
  $\e = \eone,\,\ethree = \etwo$
  %
  % \pstep def(2)
  \case{\e \pstep \eone \textrm{ by } \pstep def(2)} 
  By the fact that $\e$ is a $\betaneednr$ redex, size lemma(\ref{lem:size}), 
  and IH (same reasoning as redex text).
  %
  % \pstep def(3)
  \case{\e \pstep \eone \textrm{ by } \pstep def(3)} 
  $\e = \ap{\e_1}{\e_2},$ \\
  $\eone = \ap{\e_1'}{\e_2'},$ \\
  $\e_1 \pstep \e_1'$, $\e_2 \pstep \e_2'$ \\
   By subcases of $\eone \srstep \etwo$:
  \begin{byCases}
  \renewcommand{\byCasesCaseTemplate}[1]{\textbf{Subcase {#1}.}}
    %
    % Subcase: e* is betaneed redex
    \case{\eone \textrm{ is a } \betaneednr \textrm{ redex}}
    $\eone = \inAp{\ap{\inA[_1]{\lx\inAm{\Ex}}}{\Av[_2]}},$ \\
    $\etwo = \inAp{\inA[_1]{\inA[_2]{\substx{\inAm{\Ex}}{\val}}}}$ \\
    (Similar reasoning to Case 3.4 below)
    \begin{byCases}
      \renewcommand{\byCasesCaseTemplate}[1]{\textbf{Subsubcase {#1}.}}
      %
      % Subsubcase: \Ap = \hole
      \case{\Ap = \hole}
      $\Am = \hole,\,\e_1 \pstep \inA[_1]{\lx\Ex},\,\e_2 \pstep\Av[_2]$\\
      By lemma~\ref{lem:srstepans} and $\Ex$ shape lemma,
      there exists $\inA[_1']{\lx\Ex[']}$ and $\inA[_2']{\val'}$ s.t. \\
      $\underline{ \e_1 \srsteps \inA[_1']{\lx\Ex[']} 
                        \pstep \inA[_1]{\lx\Ex}}$ and \\
      $\underline{ \e_2 \srsteps \inA[_2']{\val'} \pstep \Av[_2] }$. Thus,
      %
      \begin{align*}
        \e = \ap{\e_1}{\e_2} &\srsteps \ap{\inA[_1']{\lx\Ex[']}}{\e_2} \\
                            &\srsteps \ap{\inA[_1']{\lx\Ex[']}}
                                         {\inA[_2']{\val'}} \\
                            &\pstep \ap{\inA[_1]{\lx\Ex}}{\inA[_2]{\val}}
      \end{align*}
      so $\ethree=\ap{\inA[_1']{\lx\Ex[']}}{\inA[_2']{\val'}}$
      %
      % Subsubcase: \Ap = \hole
      \case{\Ap = \ap{\inA[_3]{\Ap_1}}{\e_2'}}
      Then $\Am = \inA[_4]{\ly\Am_1}$ s.t. $\inAp[_1]{\Am_1} \in \A$,
      so $\e_1 \pstep 
      \inA[_3]{\inAp[_1]{\ap{\inA[_1]{\lx\inA[_4]{\ly\inAm[_1]{\Ex}}}}
                           {\Av[_2]}}},$
      which is an answer, bc there is now one more $\lambda$ than arguments.
      \begin{byCases}
        \renewcommand{\byCasesCaseTemplate}[1]{\textbf{Subsubsubcase {#1}.}}
        %
        % Subsubsubcase: lambda y is the innermost, unpaired lambda
        \case{\Ap_1 = \Am_1 = \hole,\, (\ly \textrm{ is innermost lambda})}
        $\e_1 \pstep 
        \inA[_3]{\ap{\inA[_1]{\lx\inA[_4]{\ly\inE{x}}}}{\Av[_2]}} = 
        \inA[_5]{\ly\inE{x}}$, 
        where $\A_5 = \inA[_3]{{\ap{\inA[_1]{\lx\A_4}}{\Av[_2]}}}$\\
        By lemma~\ref{lem:srstepans}, and $\Ex$ shape lemma, there exists 
        $\A_5'$ and $\Ex[']$ s.t. \\
        $\underline{\e_1 \srsteps \inA[_5']{\ly\Ex[']} 
                         \pstep \inA[_5]{\ly\Ex}}$. \\
        Let $\underline{ \A_5' = \inA[_3']{{\ap{\inA[_1']{\lx\A_4'}}
                                              {\e_3}}} }$ s.t. 
        $\e_3 \pstep \Av[_2]$,\\
        so by lemma~\ref{lem:srstepans}, there is some $\inA[_2']{\val'}$ s.t. 
        $\underline{ \e_3 \srsteps \inA[_2']{\val'} \pstep\Av[_2] }$. Thus,
        \begin{align*}
          \e = \ap{\e_1}{\e_2} 
            &\srsteps \ap{\inA[_5']{\ly\Ex[']}}{\e_2} \\
            &=\ap{\inA[_3']{{\ap{\inA[_1']{\lx\inA[_4']{\ly\Ex[']}}}{\e_3}}}}
                 {\e_2}\\
            &\srsteps\ap{\inA[_3']{{\ap{\inA[_1']{\lx\inA[_4']{\ly\Ex[']}}}
                                       {\inA[_2']{\val'}}}}}{\e_2}\\
            &\pstep \ap{\inA[_3]{{\ap{\inA[_1]{\lx\inA[_4]{\ly\inE{x}}}}
                                     {\Av[_2]}}}}{\e_2'}
        \end{align*}
        So $\ethree = 
        \ap{\inA[_3']{{\ap{\inA[_1']{\lx\inA[_4']{\ly\Ex[']}}}
                          {\inA[_2']{\val'}}}}}{\e_2'}$
        %
        % Subsubsubcase: lambda z is innermost unpaired lambda
        \case{otherwise}
        Let $\Am_1 = \inAm[_2]{\lz\hole}$ s.t. $\lz$ is the innermost unpaired 
        $\lambda$ so that \\
        $\e_1 \pstep 
        \inA[_3]{\inAp[_1]{{\ap{\inA[_1]{\lx\inA[_4]{\ly\inAm[_2]{\lz\inE{x}}}}}
                               {\Av[_2]}}}}$\\
        Apply lemma~\ref{lem:srstepans} to $\e_1 \pstep \inA[_5]{\lz\Ex}$,
        $\A_5 = 
        \inA[_3]{\inAp[_1]{{\ap{\inA[_1]{\lx\inA[_4]{\ly\Am_2}}}
                               {\Av[_2]}}}}$
        and repeat analysis of above subcase.
      \end{byCases}
    \end{byCases}
    %
    \mbox{\hspace{-1cm}
          \textit{(remaining subcases check possible} $\E$\textit{s)}}
    %
    % Subcase: e* reduction in E e context
    \case{\eone = \ap{\inE{\e_3}}{\e_2'},\,\etwo = \ap{\inE{\e_3'}}{\e_2'},\,
          \e_3 \betaneednr \e_3'} 
    By IH on $\e_1 \pstep \inE{\e_3} \srstep \inE{\e_3'}$.
    %
    % Subcase: e* reduction in A[E] context
    \case{\eone = \inA{\inE{\e_3}},\,\etwo = \inA{\inE{\e_3'}},\,
          \e_3 \betaneednr \e_3'}
    If $\A = \hole$, check cases of $\E$.\\
    If $\A = \ap{\inA[_1]{\lx\A_2}}{\e_2'}$, then
    $\eone = \ap{\inA[_1]{\lx\inA[_2]{\inE{\e_3}}}}{\e_2'}$, 
    where $\e_1 \pstep \inA[_1]{\lx\inA[_2]{\inE{\e_3}}}$\\
    By lemma~\ref{lem:srstepans}, there exists $\inA[_1']{\lx\e_4}$ s.t.
    $\underline{\e_1 \srsteps \inA[_1']{\lx\e_4} 
                     \pstep \inA[_1]{\lx\inA[_2]{\inE{\e_3}}}}$,\\
    where $\A_1' \pstep \A_1$ and $\e_4 \pstep \inA[_2]{\inE{\e_3}}$\\
    Since $\e_4 \pstep \inA[_2]{\inE{\e_3}} \srstep \inA[_2]{\inE{\e_3'}}$,
    by IH, there exists $\e_5$ s.t. 
    $\underline{ \e_4 \srsteps \e_5 \pstep \inA[_2]{\inE{\e_3'}} }$, Thus,
    \begin{align*}
    \e = \ap{\e_1}{\e_2} &\srsteps \ap{\inA[_1']{\lx\e_4}}{\e_2} \\
                        &\srsteps \ap{\inA[_1']{\lx\e_5}}{\e_2} \\
                        &\pstep \ap{\inA[_1]{\lx\inA[_2]{\inE{\e_3'}}}}{\e_2'}
    \end{align*}
    So $\ethree = \ap{\inA[_1']{\lx\e_5}}{\e_2}$
    %
    % Subcase: e* reduction in arg context
    \case{\eone = \inAp{\ap{\inA{\lx\inAm{\inE[_1]{x}}}}{\inE[_2]{\e_3}}},\,
          \etwo = \inAp{\ap{\inA{\lx\inAm{\inE[_1]{x}}}}{\inE[_2]{\e_3'}}},\,
          \e_3 \betaneednr \e_3', \inAp{\Am}\in\A}
    (Similar reasoning to case 3.1 above)
    \begin{byCases}
      \renewcommand{\byCasesCaseTemplate}[1]{\textbf{Subsubcase {#1}.}}
      %
      % Subsubcase: Ap = hole
      \case{\Ap = \hole}
    If $\Ap = \hole$, then $\Am = \hole$, so
    $\e_1 \pstep \inA{\lx\inE[_1]{x}}$ and $\e_2 \pstep \inE[_2]{\e_3}$\\
    Since $\e_1 \pstep \inA{\lx\inE[_1]{x}}$, by lemma~\ref{lem:srstepans}, 
    there exists $\A'$ and $\e_5$ s.t. 
    $\underline{\e_1 \srsteps \inA[']{\lx\e_5} \pstep \inA{\lx\inE[_1]{x}}}$,\\
    where $\A' \pstep \A$ and $\e_5 \pstep \Ex[_1]$. By the $\Ex$ shape lemma, 
    $\underline{\e_5 = \Ex[_1']}$, where $\E_1' \pstep \E_1$.\\
    Since $\e_2 \pstep \inE[_2]{\e_3} \srstep \inE[_2]{\e_3'}$, by IH,
    there exists $\e_4$ s.t. $\underline{\e_2 \srsteps \e_4
                                             \pstep \inE[_2]{\e_3'}}$. Thus,
    \begin{align*}
      \e = \ap{\e_1}{\e_2} &\srsteps \ap{\inA[']{\lx\Ex[_1']}}{\e_2} \\
                           &\srsteps \ap{\inA[']{\lx\Ex[_1']}}{\e_4} \\
                           &\pstep \ap{\inA{\lx\Ex[_1]}}{\inE[_2]{\e_3'}}
    \end{align*}
    so $\ethree = \ap{\inA[']{\lx\Ex[_1']}}{\e_4}$.
    %
    % Subsubcase: Ap \neq hole
    \case{\Ap = \ap{\inA[_1]{\Ap_1}}{\e_2'}}
    Then $\Am = \inA[_2]{\ly\Am_1}$ s.t. $\inAp[_1]{\Am_1} \in \A$,
    so $\e_1 \pstep 
    \inA[_1]{\inAp[_1]{{\ap{\inA{\lx\inA[_2]{\ly\inAm[_1]{\inE[_1]{x}}}}}
                          {\inE[_2]{\e_3}}}}}$, 
    which is an answer, bc there is now one more $\lambda$ than arguments.
    \begin{byCases}
      \renewcommand{\byCasesCaseTemplate}[1]{\textbf{Subsubsubcase {#1}.}}
      %
      % Subsubsubcase: ly is innermost lambda
      \case{\Ap_1 = \Am_1 = \hole,\, (\ly \textrm{ is innermost lambda})}
    $\e_1 \pstep 
    \inA[_1]{{\ap{\inA{\lx\inA[_2]{\ly\inE[_1]{x}}}}{\inE[_2]{\e_3}}}} = 
    \inA[_3]{\ly\inE[_1]{x}}$, 
    where $\A_3 = \inA[_1]{{\ap{\inA{\lx\A_2}}{\inE[_2]{\e_3}}}}$\\
    By lemma~\ref{lem:srstepans}, and $\Ex$ shape lemma, there exists 
    $\A_3'$ and $\Ex[_1']$ s.t. \\
    $\underline{\e_1 \srsteps \inA[_3']{\ly\Ex[_1']} 
                     \pstep \inA[_3]{\ly\Ex[_1]}}$ \\
    Let $\underline{ \A_3' = \inA[_1']{{\ap{\inA[']{\lx\A_2'}}{\e_4}}} }$ s.t. 
    $\e_4 \pstep \inE{\e_3} \srstep \inE[_2]{\e_3'}$,\\
    so by IH, there is some $\e_5$ s.t. 
    $\underline{ \e_4 \srsteps \e_5 \pstep\inE[_2]{\e_3'} }$. Thus,
    \begin{align*}
    \e = \ap{\e_1}{\e_2} &\srsteps \ap{\inA[_3']{\ly\Ex[_1']}}{\e_2}\\
 &=\ap{\inA[_1']{{\ap{\inA[']{\lx\inA[_2']{\ly\Ex[_1']}}}{\e_4}}}}{\e_2}\\
 &\srsteps\ap{\inA[_1']{{\ap{\inA[']{\lx\inA[_2']{\ly\Ex[_1']}}}{\e_5}}}}{\e_2}\\
 &\pstep 
 \ap{\inA[_1]{{\ap{\inA{\lx\inA[_2]{\ly\inE[_1]{x}}}}{\inE[_2]{\e_3'}}}}}{\e_2'}
  \end{align*}
    So $\ethree = 
       \ap{\inA[_1']{{\ap{\inA[']{\lx\inA[_2']{\ly\Ex[_1']}}}{\e_5}}}}{\e_2}$
       %
       % Subsubsubcase: lz is innermost lambda
    \case{otherwise}
    Let $\Am_1 = \inAm[_2]{\lz\hole}$ s.t. $\lz$ is the innermost unpaired 
    $\lambda$ so that \\
    $\e_1 \pstep 
    \inA[_1]{\inAp[_1]{{\ap{\inA{\lx\inA[_2]{\ly\inAm[_2]{\lz\inE[_1]{x}}}}}
          {\inE[_2]{\e_3}}}}}$\\
    Apply lemma~\ref{lem:srstepans} to $\e_1 \pstep \inA[_3]{\lz\Ex[_1]}$ and
    repeat analysis of above subcase.
    \end{byCases}
    \end{byCases}
  \end{byCases}
  %
  % \pstep def(4)
  \case{\e \pstep \e' \textrm{ by } \pstep def(4)}
  Impossible because not a $\betaneednr$ redex.
\end{byCases}
\end{proof}





%
% srstep Answer lemma
% if e => A[\x.e'], there exists A' and e'' s.t. e |-->> A'[\x.e''] => A[\x.e']
%
\begin{lemma}
\label{lem:srstepans}$\;$
\begin{enumerate}
\item If $\e \pstep \inA{\lxe'}$, there exists $\A'$ and $\e''$ s.t. $\e
  \srsteps \inA[']{\lx\e''} \pstep \inA{\lx\e'}$, where $\A' \pstep \A$ and
  $\e'' \pstep \e'$.
\item If $\e \pstep x$, then $\e \srsteps x$.
\end{enumerate}
\end{lemma}
\begin{proof}
Same reasoning as redex text.
\end{proof}





\pagebreak



\newcommand{\sizesubstx}[2]{\size{\substx{#1}{\val} \pstep \substx{#2}{\val'}}}
\newcommand{\sizeRHS}[2]
           {\size{#1 \pstep #2} + \numfvx{#2} \times \size{\val \pstep \val'} }

\newcommand{\sizeIH}[2]{ \sizesubstx{#1}{#2} \leq \sizeRHS{#1}{#2} }

%
% Size Lemma
%
\begin{lemma}[Size]
\label{lem:size}
If $s_\e = \size{\e \pstep \e'}$ and $s_\val = \size{\val \pstep \val'}$ then:
\begin{enumerate}
\item $\substx{\e}{\val} \pstep \substx{\e'}{\val'}$
\item $\sizeIH{\e}{\e'}$
\end{enumerate}
\end{lemma}

\renewcommand{\eone}{\inAp{\ap{\inA[_1]{\ly\inAm{\Ey}}}{\inA[_2]{\val_1}}}}
\renewcommand{\etwo}{\inAp[']{\inA[_1']{\inA[_2']{\substy{\inAm[']{\Ey[']}}{\val_1'}}}}}
\begin{proof}
By induction on $\e$. Case analysis on last step in $\e \pstep \e'$.
\begin{byCases}
  \case{\e \pstep \e' \textrm{ by } \pstep def(1)}
  $\e = \e'$. Proof by subcases on $\e$. Same reasoning as redex book.
  \case{\e \pstep \e' \textrm{ by } \pstep def(2)}
  $\e = \inAp{\ap{\inA[_1]{\ly\inAm{\Ey}}}{\inA[_2]{\val_1}}},$\\
  $\e' = \inAp[']{\inA[_1']{\inA[_2']{\substy{\inAm[']{\Ey[']}}{\val_1'}}}}$\\
  $\size{\e \pstep \e'} = \size{\Ap \pstep \Ap'} + \size{\A_1 \pstep \A_1'} +
    \size{\A_2 \pstep A_2'} + \size{\inAm{\Ey} \pstep \inAm[']{\Ey[']}} + 
    \numfv{y}{\inAm[']{\Ey[']}} \times \size{\val_1 \pstep \val_1'} + 1$ \\
  By IH:\\
  $\sizeIH{\Ap}{\Ap'}$ \\
  $\sizeIH{\A_1}{\A_1'}$ \\
  $\sizeIH{\inAm{\Ey}}{\inAm[']{\Ey[']}}$ \\
  $\sizeIH{\A_2}{\A_2'}$ \\
  $\sizeIH{\val_1}{\val_1'}$ 
  \begin{align*}
    &\size{\substx{\e}{\val} \pstep \substx{\e'}{\val'}}  \\
    =\; &\sizesubstx{\Ap}{\Ap'} + \sizesubstx{\A_1}{\A_1'} + 
         \sizesubstx{\A_2}{\A_2'} + \\
        &\sizesubstx{\inAm{\Ey}}{\inAm[']{\Ey[']}} + \\
        &\numfvy{\inAm[']{\Ey[']}} \times \sizesubstx{\val_1}{\val_1'} + 1 \\
 \leq\; &\sizeRHS{\Ap}{\Ap'} + \\ 
        &\sizeRHS{\A_1}{\A_1'} + \\
        &\sizeRHS{\A_2}{\A_2'} + \\
        &\sizeRHS{\inAm{\Ey}}{\inAm[']{\Ey[']}} + \\
        &\numfvy{\inAm[']{\Ey[']}} \times (\sizeRHS{\val_1}{\val_1'}) + 1 \\
    =\; &\sizeRHS{\Ap}{\Ap'} + \\
        &\sizeRHS{\A_1}{\A_1'} + \\
        &\sizeRHS{\A_2}{\A_2'} + \\
        &\sizeRHS{\inAm{\Ey}}{\inAm[']{\Ey[']}} + \\
        &\numfvy{\inAm[']{\Ey[']}} \times \size{\val_1 \pstep \val_1'} + 
         \numfvy{\inAm[']{\Ey[']}} \times \numfvx{\val_1'} \times
         \size{\val \pstep \val'} + 1 \\
    =\; &\size{\Ap\pstep\Ap'} + \size{\A_1\pstep\A_1'} + \size{\A_2\pstep\A_2'}+
         \size{\inAm{\Ey}\pstep\inAm[']{\Ey[']}} + 
         \numfvy{\inAm[']{\Ey[']}} \times \size{\val_1 \pstep \val_1'} + 1 + \\
         &(\numfvx{\Ap'}+\numfvx{\A_1'}+\numfvx{\A_2'}+
          \numfvx{\inAm[']{\Ey[']}}+
          \numfvy{\inAm[']{\Ey[']}} \times \numfvx{\val_1'}) 
         \times \size{\val \pstep \val'} \\
    =\; & \size{\e \pstep \e'} + \numfvx{\e'} \times \size{\val \pstep \val'}
  \end{align*}
  \case{\e = \ap{\e_1}{\e_2}} Same reasoning as redex book.
  \case{\e = \lx\e_1} Same reasoning as redex book.
\end{byCases}
\end{proof}






\pagebreak




\begin{lemma}[Unique Decomposition]
  For all terms $\e$ and variables $x$, if $\e \neq \Ex[']$, $x$ not bound by
  $\E'$, then $\e$ is either an answer $\Av$, or there exists a unique
  evaluation context $\E$ and \betaneednr redex $\e'$ such that $\e =
  \inE{\e'}$.
\end{lemma}
%
\begin{proof}
  By structural induction on $\e$.
\begin{byCases}
  %
  % case 1: e = x
  %
  \case{\e = x} Impossible because $\e = \Ex$, $\E = \hole$
  %
  % case 2: e = \x.e
  %
  \case{\e = \lxe_1} 
  Claim holds with $\e$ an answer, $\A = \hole$, $\val = \e$.
  %
  % case 3: e =  e e
  %
  \case{\e = \ap{\e_1}{\e_2}}
  $\e_1 \neq \Ex['']$, otherwise, $\e = \Ex[']$, where $\E' = \ap{\E''}{\e_2}$.
  By IH, $\e_1$ is either an answer or uniquely decomposes, so
  proceed by subcases on $\e_1$.
  \begin{byCases}
    \renewcommand{\byCasesCaseTemplate}[1]{\textbf{Subcase {#1}.}}  
    %
    % subcase 1: e_1 = E[e]
    %
    \case{\e_1 = \inE[_1]{\e_3}}
    By IH, $\e_3$ is a \betaneednr redex. Then the claim holds with 
    $\E = \ap{\E_1}{\e_2}$ and $\e' = \e_3$.
    %
    % subcase 2: e_1 = A[v]
    %
    \case{\e_1 = \inA[_1]{\val_1}}
    Let $\val_1 = \lxe_3$. Proceed by subsubcases on $\e_3$.
    \begin{byCases}
      \renewcommand{\byCasesCaseTemplate}[1]{\textbf{Subsubcase {#1}.}}
      %
      % subsubcase 1: e_3 = E[y]
      %
      \case{\e_3 = \Ey[_1]} 
      where $y$ is a variable bound by the answer context
      $\ap{\inA[_1]{\lx\hole}}{\e_2}$.
      Proceed by subsubsubcases on $\e_2$
      \begin{byCases}
        \renewcommand{\byCasesCaseTemplate}[1]{\textbf{Subsubsubcase {#1}.}}
        \case{\e_2 = \inE[_2]{\e_4}}
        $\e_4$ is a \betaneednr redex. Then the claim holds with 
        $\E = \ap{\inA[_1]{\lx\Ey[_1]}}{\E_2}$ and $\e' = \e_4$.
        \case{\e_2 = \inA[_2]{\val_2}}
        Then $\e$ is a \betaneednr redex.
      \end{byCases}
      %
      % subsubcase 1: e_3 \neq E[y] = E[e]
      %
      \case{\e_3 \neq \Ey[_1] = \inE[_2]{\e_4}}
      $\e_4$ is a \betaneednr redex. 
      Claim holds with $\E = \ap{\inA[_1]{\lx\inE[_2]{\,\,\,}}}{\e_2}$ and 
      $\e' = \e_4$.
      %
      % subsubcase 1: e_3 \neq E[y] = A[v]
      %
      \case{\e_3 \neq \Ey[_1] = \inA[_2]{\val_2}}
      Claim holds where $\e$ is an answer with 
      $\A = \ap{\inA[_1]{\lx\inA[_2]{\,\,\,}}}{\e_2}$ and $\val = \val_2$
    \end{byCases}
  \end{byCases}
\end{byCases}
\end{proof}

\end{document}
